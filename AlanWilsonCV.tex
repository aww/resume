%% start of file `template_en.tex'.
%% Copyright 2007 Xavier Danaux (xdanaux@gmail.com).
%
% This work may be distributed and/or modified under the
% conditions of the LaTeX Project Public License version 1.3c,
% available at http://www.latex-project.org/lppl/.


\documentclass[10pt,letter]{moderncv_new}

% moderncv themes
\moderncvtheme[blue]{classic}
%\moderncvtheme[blue]{casual}                 % optional argument are 'blue' (default), 'orange', 'red', 'green', 'grey' and 'roman' (for roman fonts, instead of sans serif fonts)
%\moderncvtheme[green]{classic}                % idem

% character encoding
\usepackage[utf8]{inputenc}                   % replace by the encoding you are using

% adjust the page margins
%\usepackage[scale=0.8]{geometry}
%\recomputelengths                             % required when changes are made to page layout lengths

% personal data
\firstname{Alan}
\familyname{Wilson}
\title{Experimental Physics (HEP)}               % optional, remove the line if not wanted
\address{106 Burwood Ave \#2}{Ann Arbor, MI 48103}    % optional, remove the line if not wanted
\mobile{(734) 239-3309}
%\phone{phone (optional)}
%\fax{fax (optional)}
\email{aww@umich.edu}
\extrainfo{http://cern.ch/wilsona}
%\photo[64pt]{picture}                         % '64pt' is the height the picture must be resized to and 'picture' is the name of the picture file; optional, remove the line if not wanted
%\quote{Some quote (optional)}

%\nopagenumbers{}                             % uncomment to suppress automatic page numbering for CVs longer than one page

%----------------------------------------------------------------------------------
%            content
%----------------------------------------------------------------------------------
\begin{document}
%\renewcommand{\listitemsymbol}{+}

\maketitle

\section{Education}
\cventry{2010}{Ph.D. Physics}{University of Michigan}{Ann Arbor}{}{}
\cventry{2003}{M.S. Mathematics}{University of Michigan}{Ann Arbor}{}{}
\cventry{1999}{B.S. Mathematical Sciences (Physics)}{University of Washington}{Seattle}{}{}
\cventry{1999}{B.S. Computer Engineering}{University of Washington}{Seattle}{}{}
%\cventry{Year}{Degree}{Institution}{City}{\textit{Grade}}{Description}  % arguments 3 to 6 are optional

\section{Ph.D. thesis}
\cvline{title}{\emph{Search for a Supersymmetry Signature with the $Z\gamma$ plus Missing Transverse Energy Final State Using the D\O\ Detector}}
\cvline{supervisor}{Prof. Bing Zhou}
\cvline{description}{\small
Over the last four years the D\O\ detector at Fermilab's Tevatron Collider has collected 
$6.2$~fb$^{-1}$ of 1.96~TeV proton-antiproton collisions.
Using this data I searched for a supersymmetry signature: a high-$p_T$ photon,
a pair of electrons or muons from a $Z$ boson decay,
and large missing transverse energy produced by undetected particles.
Certain gauge mediated supersymmetry breaking (GMSB)
models would produce $Z\gamma$ events with large transverse missing energy,
but this is the first time that this signature has been searched for in a hadron collider experiment.
The dominate background is from standard model $Z\gamma$ production,
these events are used as a control sample for data-Monte Carlo (MC) consistency checks.
Background estimates are accomplished using data-driven methods and MC simulations.
The observed data is consistent with standard model predictions.
No evidence for a supersymmetric signal is found.
Using the extra signal discrimination obtained with boosted decision trees (a trained multivariate selection technique)
and combining the analysis results from both electron and muon channels
we exclude our GMSB model at the 95\% confidence level for $70 < \Lambda < 117.5$~TeV.
This corresponds to neutralino (the next-to-lightest superparticle) masses of
$111 < \tilde{\chi}_1^0 < 222$~GeV.
}

\section{Experience}
\subsection{Research \& Hardware Experience}
\cventry{2009--present}{Graduate Student Research Assistant}{D\O\ Experiment}{}{}{}
\cvlistitem{Searched for supersymmetry with the $Z\gamma$+missing $E_T$ signature.}
\cvlistitem{Collaborating on standard model $Z\gamma$ cross section and anomalous couplings.}
\cvlistitem{Managing Monte Carlo production for the new phenomena group.}
\cvlistitem{Control rooms shifts for data acquisition (DAQ).}
\cventry{2005--2008}{Graduate Student Research Assistant}{ATLAS Experiment}{}{}{}
\cvlistitem{Primary contributor to diboson physics: prepared for cross-section measurement and estimated sensitivity to anomalous couplings.}
\cvsublistitem{Contributed entire $WZ$ analysis and substantial portions of $WW$ and $ZZ$.}
\cvsublistitem{Worked with MC@NLO and BosoMC for event generation, cross-section systematics, and modeling of anomalous couplings.}
\cvsublistitem{Developed calculation of the confidence region for couplings.}
\cvlistitem{Constructed the gas monitor chamber for the monitored drift tube (MDT) system:}
\cvsublistitem{Working with an engineer and an undergraduate,
  assembled a complete table-top (6 layers of 16 tubes)
  drift tube detector from scratch including scintillator triggers and gas system.}
\cvsublistitem{Used to monitor the gas mixture flowing in and out of the ATLAS MDT system
  via features of the drift-time spectra from cosmic ray muon tracks.}
\cvlistitem{Contributions to the $H\rightarrow WW$ analysis work.}
\cvlistitem{Applying boosted decision trees to electron identification and b-tagging.}
\cvsublistitem{Testing of algorithm and integration into ATLAS software.}
\cventry{2004--2005}{Research Assistant}{ATLAS Experiment}{Univ. of Michigan}{Geneva, Switzerland}{}
\cvlistitem{Phase 1 commissioning of MDT detectors (for the muon precision measurement).}
\cvsublistitem{Setup lab space from scratch: trained to operate cranes,
  supervised chamber movements, expert in the gas mixing and distribution system}
\cvsublistitem{Lead a team of undergraduates (as many of as five) finishing chamber assembly,
  leak checking, commissioning with cosmic rays, and working out repairs on the fly}
\cvlistitem{Validating muon reconstruction software}
\cvsublistitem{Compared muon reconstruction algorithms for resolution and efficiency in MC}
\cventry{1994--1999}{Research Assistant}{Space Sciences, Geophysics}{Univ. of Washington}{Seattle}{}
\cvlistitem{Assembled DAQ hardware and developed software for balloon and satellite experiment commissioning.}
\cvlistitem{Research and simulations exploring coded aperture imaging using X-rays.}

\subsection{Teaching}
\cventry{1999--2003}{Graduate Student Instructor}{Mathematics}{University of Michigan}{Ann Arbor}{Courses: precalculus, calculus, and differential equations}
\cventry{1998--1999}{Teaching Assistant}{Computer Science and Engineering}{Univ. of Washington}{Seattle}{Courses: Discrete Structures, Introduction to Computer Graphics, and Digital System Design}

\section{Conference Talks and Publications}
\cvline{D0 note}{\small ``Search for supersymmetry with the $Z\gamma$ plus missing transverse energy signature'', in preparation}
\cvline{publication}{\small ``The ATLAS Experiment at the CERN Large Hadron Collider.'' \textit{JINST} 3 S08003 (2008)}
\cvline{publication}{\small ``Expected Performance of the ATLAS Experiment - Detector, Trigger and Physics.'' CERN-OPEN-2008-020 (2009), arXiv:0901.0512}
\cvline{ATLAS note}{\small ``Diboson physics studies with the ATLAS detector'', ATL-PHYS-PUB-2009-038\newline{}\textit{Expanded version of ``Expected Performance....''}}
\cvline{proceedings}{\small ``Diboson Physics in ATLAS'', ATL-PHYS-PROC-2008-051}
\cvline{talk}{\small ``Direct and Indirect Searches for New Physics with Diboson Final States ,'' LHC New Physics Signatures Workshop, University of Michigan, January 2008.}
\cvline{publication}{\small ``Drift time spectrum and gas monitoring in the ATLAS Muon Spectrometer precision chambers.'' Nucl.\ Instrum.\ Methods A \textbf{588}, 347 (2008).}
\cvline{proceedings}{\small ``Streamlined Calibration of the ATLAS Muon Spectrometer Precision Chambers'', D. Levin for the ATLAS Collaboration, Nuclear Science Symposium Conf. Record, 2009 IEEE, 1040--1044}
\cvline{proceedings}{\small ``Gas performance of the ATLAS MDT precision chambers'', Nuclear Science Symposium Conference Record, 2008. NSS '08. IEEE, 19-25 Oct. 2008, 3213}
\cvline{publication}{\small ``A Multivariate Training Technique with Event Reweighting.'' The ATLAS Collaboration, JINST3:P04004,2008 }
\cvline{ATLAS note}{\small ``B-tagging Based on Boosted Decision Trees and Performance Comparisons of ATLAS B-taggers'', ATL-PHYS-INT-2009-072}
\cvline{ATLAS note}{\small ``Performance of Electron Identification Based on Boosted Decision Trees'', ATL-COM-PHYS-2009-160}
\cvline{ATLAS note}{\small ``Study of the SM Higgs Discovery Potential through W-pair Leptonic Decay Modes with Boosted Decision Trees'', ATL-PHYS-INT-2009-052}
\cvline{talk}{\small ``$Z\rightarrow\mu\mu$ performance'', ATLAS Physics Workshop, Rome, June 2005. }
\subsection{ATLAS general}
\cvline{publication}{\small ``Charged-particle multiplicities in $pp$ interactions at $\sqrt{s}=900$~GeV measured with the ATLAS detector at the LHC'', ATLAS Collaboration, Phys. Lett. B 688, issue 1, 21 (2010).}
\cvline{publication}{\small ``Search for New Particles in Two-jet Final States in 7 TeV Proton-Proton Collisions with the ATLAS Detector at the LHC'', ATLAS Collaboration, Phys. Ref. Lett. 105, 161801 (2010)}
\cvline{publication}{\small ``Measurement of the $W\rightarrow l\nu$ and $Z/\gamma^* \rightarrow \ell\ell$ production cross sections in proton-proton collisions at $\sqrt{s}=7$~TeV with the ATLAS Detector'', ATLAS Collaboration, accepted by JHEP (2010).}
\cvline{publication}{\small ``Observation of a centrality-dependent dijet asymmetry in lead-lead collisions at $\sqrt{s_{NN}}=2.76$~TeV with the ATLAS detector at the LHC'', ATLAS Collaboration, Phys. Ref. Lett. 105, 252303 (2010).}




%\closesection{}                   % needed to renewcommands
%\renewcommand{\listitemsymbol}{-} % change the symbol for lists

%\section{Extra 1}
%\cvlistitem{Item 1}
%\cvlistitem{Item 2}
%\cvlistitem[+]{Item 3}            % optional other symbol

%\section{Extra 2}
%\cvlistdoubleitem[\Neutral]{Item 1}{Item 4}
%\cvlistdoubleitem[\Neutral]{Item 2}{Item 5}
%\cvlistdoubleitem[\Neutral]{Item 3}{}

% Publications from a BibTeX file
%\nocite{*}
%\bibliographystyle{plain}
%\bibliography{publications}       % 'publications' is the name of a BibTeX file

\end{document}


%% end of file `template_en.tex'.
