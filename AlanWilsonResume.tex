\documentclass[11pt,letterpaper,sans]{moderncv}

\moderncvtheme{classic}
\moderncvcolor{blue}

\usepackage[utf8]{inputenc}
\usepackage[scale=0.87]{geometry}

% For some reason the {\color{color1} #1} version of this
% leads to space before a list that starts with \hilight
\newcommand{\hilight}[1]{\textcolor{color1}{\itshape#1}}

\firstname{Alan}
\familyname{WILSON}
\title{Ph.D., experimental high energy physics}
\address{310 Rue des Hautains de la Crotte}{01210 Ornex}{France}
\mobile{+41 76 487 4147}
\phone{+1 734 239-3309}
\email{alan.w.wilson@gmail.com}
\homepage{http://cern.ch/wilsona}

%\nopagenumbers{}

\begin{document}
%\renewcommand{\listitemsymbol}{+}

\makecvtitle

\section{The basics}
\cvitem{}{As a physicist and a curious human,
  everyday I collaborate with colleagues to identify and understand structure in data
  and communicate our findings widely.}

\section{Education}
\cventry{2011}{Ph.D. Physics}{University of Michigan}{Ann Arbor}{}{}
\cventry{2003}{M.S. Mathematics}{University of Michigan}{Ann Arbor}{}{}
\cventry{1999}{B.S. Computer Engineering}{University of Washington}{Seattle}{}{}
\cventry{1999}{B.S. Mathematical Sciences}{University of Washington}{Seattle}{}{}

%%%%%%%%%%%%%%%%%%%%%%%%%%%%%%%%%%%%%%%%%%%%%%%%%%%%%%%%%%%%%%%%%%%%%%%%%%%%%%%
\section{Proficiencies \& technical interests}
%%%%%%%%%%%%%%%%%%%%%%%%%%%%%%%%%%%%%%%%%%%%%%%%%%%%%%%%%%%%%%%%%%%%%%%%%%%%%%%

\begin{cvcolumns}{programming}
  \cvcolumn{Almost every day}{
    \begin{itemize}
    \item C++
    \item Python
    \item ROOT+RooStats+TMVA
    \item numpy+scipy+matplotlib
  \end{itemize}
  }
  \cvcolumn{Occasionally}{
    \begin{itemize}
    \item SQL variants
    \item Mathematica and Matlab
    \item shell scripting
    \item C \& ASM for $\mu$-controllers
    \end{itemize}
  }
  \cvcolumn{Dabble in or dated}{
    \begin{itemize}
    \item Javascript
    \item Lisp dialects
    \item PHP
    \item Perl
    \end{itemize}
  }
\end{cvcolumns}

\cvitemwithcomment{command line}{git/svn, tmux/screen, emacs, ssh/rsync, etc.}{...the usual Linux/dev. stuff}
\cvitemwithcomment{statistics}{fitting, using likelihoods, Bayesian vs. frequentist, etc.}{...for quantifying level of knowledge}
\cvitemwithcomment{\footnotesize machine learning}{supervised learning, boosted decision trees (BDTs)}{...including a paper on BDTs and weights}
\cvitemwithcomment{publishing}{\small \LaTeX, PowerPoint, HTML/CSS, Photoshop, Illustrator, etc.}{...with the goal to communicate effectively}
\cvitemwithcomment{\small extra projects}{network structures, coding theory, compression}{...in a variety of graduate courses}
\cvitemwithcomment{indep. study}{Andrew Ng's machine learning course, Bill Howe's data science course, etc.}{...on Coursera}


%%%%%%%%%%%%%%%%%%%%%%%%%%%%%%%%%%%%%%%%%%%%%%%%%%%%%%%%%%%%%%%%%%%%%%%%%%%%%%%
%\section{Experience}
%%%%%%%%%%%%%%%%%%%%%%%%%%%%%%%%%%%%%%%%%%%%%%%%%%%%%%%%%%%%%%%%%%%%%%%%%%%%%%%

\section{Experience}
\cventry{2011--present}{Post Doctoral Research Fellow}{ATLAS Experiment}{}{Geneva, Switzerland}{
  As part of the largest experiment in the world, I contributed to
  the Higgs discovery (specifically, via $H\rightarrow ZZ \rightarrow 4\ell$) and
  to measurements involving multiple leptons, including the rare decay $Z\rightarrow 4\ell$.
  \begin{itemize}
  \item \hilight{Wrote readable, modular, and accurate code} to run in batch (Condor) and on the Grid
    to analyze \hilight{large amounts of data}
  %\item Mentored graduate students
  \item Developed many \hilight{tools for efficiently specifying, building, and sharing plots}
  \item \hilight{Primary editor} for at least one paper as well as internal documents
  \item Developed a \hilight{framework for defining unit tests} of numerical quantities in \LaTeX{} documents
  \item Constructed event visualizations in various forms
  \item Tested new detectors as part of a hardware installation team
  \item \hilight{Controlled experiment-wide data acquisition}, reacting quickly but thoughtfully to solve faults.
  \end{itemize}
}
\cventry{2009--2010}{Graduate Student Research Assistant}{D\O\ Experiment}{}{Batavia, Illinois}{
  \begin{itemize}
  \item \textcolor{color1}{Thesis topic}: a search for new physics via the $Z(\rightarrow\ell\ell) \gamma$+missing $E_T$ final state.\\
    This is a niche topic allowing me to contribute to nearly the \hilight{complete analysis}, including
    \begin{itemize}
    \item exploring the theory and experimental sensitivity with simulation,
    \item rejecting backgrounds with BDTs and estimating with data-driven methods, and
    \item using statistics to quantify constraints on theory imposed by the observation.
    \end{itemize}
  \item \hilight{Expert role managing Monte Carlo simulation jobs}: responding to my colleagues' requests
    and translating them into tested job specifications, submitting the jobs, and monitoring the results. 
  \item DAQ shifts: online control, monitoring, and problem solving for the data taking of a large experiment
  \end{itemize}
}
\cventry{2005--2008}{Graduate Student Research Assistant}{ATLAS Experiment}{Univ. of Michigan}{Ann Arbor}{
  \begin{itemize}
  \item \hilight{Primary contributor to large public documents} on diboson physics sensitivity before data was available.
  \item Implemented tools for calculating confidence regions via \hilight{marginalized likelihoods}.
  \item Collaborating with an engineer and supervising a student,
    \hilight{constructed the gas monitor chamber} for the muon tracking system of ATLAS.
  \item Applied \hilight{boosted decision trees} to particle identification tasks (electron id. and b-tagging),
    becoming a local expert on the ATLAS software framework
  \end{itemize}
}
\cventry{2004--2005}{Research Assistant}{ATLAS Experiment}{Univ. of Michigan}{Geneva, Switzerland}{
  \begin{itemize}
  \item \hilight{Validated muon reconstruction software} with systematic comparisons, uncovering faults
  \item Commissioning of 40 large muon detectors, which involved
    \begin{itemize}
    \item \hilight{leading a team of five students} to complete assembly and testing,
      % leak checking, commissioning with cosmic rays, and working out repairs on the fly
    \item \hilight{managing logistics} of the lab space when our supervisor was away, and
    \item training to operate cranes and becoming an expert in the gas mixing and distribution system.
    \end{itemize}
  \end{itemize}
}
\cventry{1994--1999}{Research Assistant}{Space Sciences, Geophysics}{Univ. of Washington}{Seattle}{
  \begin{itemize}
  \item{Built \hilight{software testing platforms} for DAQ hardware used on balloon and satellite experiments}
  \item{\hilight{Simulated coded aperture imaging} used at X-ray wavelengths (where lenses are not possible)}
  \end{itemize}
}

\section{Teaching}
\cventry{1999--2003}{Graduate Student Instructor}{Mathematics}{University of Michigan}{Ann Arbor}{Courses: precalculus, calculus I \& II, and differential equations}
\cventry{1998--1999}{Teaching Assistant}{Computer Science and Engineering}{Univ. of Washington}{Seattle}{Courses: Discrete Structures, Introduction to Computer Graphics, and Digital System Design}


%%%%%%%%%%%%%%%%%%%%%%%%%%%%%%%%%%%%%%%%%%%%%%%%%%%%%%%%%%%%%%%%%%%%%%%%%%%%%%%
\section{Publications}
%%%%%%%%%%%%%%%%%%%%%%%%%%%%%%%%%%%%%%%%%%%%%%%%%%%%%%%%%%%%%%%%%%%%%%%%%%%%%%%

{\footnotesize
\cvline{note}{``ATLAS measurements of the 7 and 8 TeV cross sections for $Z\rightarrow 4\ell$ in pp collisions'', May 2013. ATLAS-CONF-2013-055}
\cvline{paper}{``Observation of a new particle in the search for the Standard Model Higgs boson with the ATLAS detector at the LHC'', Phys. Lett. B 716 (2012) 1-29}
\cvline{paper}{``Search for $Z\gamma$ events with large missing transverse energy in $p\bar{p}$ collisions at $\sqrt{s}=1.96$~TeV'', Phys. Rev. D 86, 071701(R) (2012)}
\cvline{publication}{``The ATLAS Experiment at the CERN Large Hadron Collider.'' \textit{JINST} 3 S08003 (2008)}
\cvline{publication}{``Expected Performance of the ATLAS Experiment - Detector, Trigger and Physics.'' CERN-OPEN-2008-020 (2009), arXiv:0901.0512}
\cvline{paper}{``Drift time spectrum and gas monitoring in the ATLAS Muon Spectrometer precision chambers.'' Nucl.\ Instrum.\ Methods A \textbf{588}, 347 (2008).}
\cvline{paper}{``A Multivariate Training Technique with Event Reweighting.'' H.-J. Yang, T. Dai, A. Wilson, Z. Zhao and B. Zhou, \href{http://dx.doi.org/10.1088/1748-0221/3/04/P04004}{JINST 3:P04004, 2008} }
\cvline{projects}{See, for instance, \url{http://cern.ch/wilsona/OtherTopics/NetworksSI708}}
}

\section{Other interests}

\cvitem{hobbies}{electronics, photography -- small analog and microcontroller projects, digital and chemical darkrooms}
\cvitem{culture}{cooking, travel, hiking, and wandering -- seeing, smelling, touching, and tasting the world}

\end{document}
