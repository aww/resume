\documentclass[11pt,letterpaper,sans]{moderncv}

\moderncvtheme{classic}
\moderncvcolor{blue}

\usepackage[utf8]{inputenc}
\usepackage[scale=0.87, top=.35in, bottom=.35in, left=.35in, right=.35in]{geometry}
% \usepackage{hyperref}
\usepackage[T1]{fontenc}
\input{glyphtounicode}

% For some reason the {\color{color1} #1} version of this
% leads to space before a list that starts with \hilight
\newcommand{\hilight}[1]{\textcolor{color1}{\itshape#1}}

\firstname{Alan}
\familyname{WILSON}
%\address{1375 Civic Center Dr}{Santa Clara 95050}{}
\mobile{(408) 242-5090}
\email{alan.w.wilson@gmail.com}
%\homepage{http://thingsstufftimes.com}
\extrainfo{\url{https://github.com/aww} \\ \url{https://www.linkedin.com/in/alanwwilson/}}

%\nopagenumbers{}

\begin{document}
%\renewcommand{\listitemsymbol}{+}

\makecvtitle

\vspace*{-8mm}

%\section{The basics}
%\cvitem{}{As a curious human,
%  I collaborate with colleagues to identify and understand structure in data
%  and communicate our findings clearly.}

%%%%%%%%%%%%%%%%%%%%%%%%%%%%%%%%%%%%%%%%%%%%%%%%%%%%%%%%%%%%%%%%%%%%%%%%%%%%%%%
%\section{Experience}
%%%%%%%%%%%%%%%%%%%%%%%%%%%%%%%%%%%%%%%%%%%%%%%%%%%%%%%%%%%%%%%%%%%%%%%%%%%%%%%

\section{Experience}

\cventry{2018--2023}{Lead Machine Learning Engineer}{Glassdoor}{}{San Francisco, CA / Remote}{
  \begin{itemize}
  \item Modernized pay estimation with deep learning and organized builds with DVC
  \item Built and maintained a bespoke model for optimal pricing in the jobs marketplace
  \item Introduced structured model testing
  \item Built machine learning training and outreach for engineers and product managers
  \end{itemize}
}

\cventry{2015--2018}{Senior Data Scientist}{Glassdoor}{}{San Francisco, CA}{
  \begin{itemize}
  \item Improved bot monitoring and flagging algorithms
  \item Built a job ad and direct response marketing demand forecasting system
  \item Built budget sizing and optimal pricing tools for job ad sales
  \item Ran an ecletic mix of small projects involving data processing and model building in offline python notebooks, hive, and some spark
  \item A/B testing platform development and maintainace (custom, in-house system):
    \begin{itemize}
    \item built out new statisical components and added a completely new and much-refined UI
    \end{itemize}
  \item Contributed many engagement and valuation insights around B2B products, CLV of marketing efforts, and user job search engagement
  \end{itemize}
}

\cventry{2014--2015}{Data Scientist}{Glassdoor}{}{San Francisco, CA}{
  \begin{itemize}
  \item Brought to production core ML/data products at Glassdoor: \textbf{categorizing jobs}, and \textbf{salary prediction}
    \begin{enumerate}
    \item End-to-end, from research and data exploration to model building and human testing to optimization for production.
    \end{enumerate}
  \item Early adopter of Tableau, company expert, author of a few very high-demand dashboards.
  \end{itemize}
}

\cventry{2014}{Fellow}{Insight Data Science}{}{Mountain View, CA}{
  \begin{itemize}
  \item Developed the web app \href{http://NewsSpectra.com}{NewsSpectra.com}
    which presents alternative news coverage of a topic on a spectrum of readability \& detail.
  \item Full stack web dev. from Web scraping, NLP processing, and web app deployment with MySQL, AWS, Bootstrap, D3, etc.
  \end{itemize}
}

\cventry{2011--2013}{Post Doctoral Research Fellow}{ATLAS Experiment}{}{Geneva, Switzerland}{
  \begin{itemize}
  \item Contributing to supersymmetry searches and the discovery of the Higgs boson
  \item Filtered massive datasets (> 1 TB) to find very rare signals using local and world-wide batch computing,
  \item Built C++ applications on top of the shared tools of a collaboration of 2000 scientists to calibrate, resolve ambiguities, and filter data.
  \item Developed a framework in Python and ROOT for specifying, building, and publishing plots to the web.
  \item Applied unit tests to publications by factoring out numerical quantities in JSON/YAML+Python.
  %\item Constructed event visualizations in various forms
  \item Controlled and monitored data acquisition, requiring quick reactions and efficient communication with colleagues.
  %\item Mentored graduate students, and was the primary editor for many documents including published papers.
  \end{itemize}
}
\cventry{2005--2010}{Graduate Student Research Assistant}{ATLAS \& D\O\ Experiments}{}{Michigan \& Illinois}{
  \begin{itemize}
  \item Used a large stack of Monte Carlo simulations (from particle production to detector response),
    as well as extrapolations from data, to quantify signal and background.
  \item Increased signal to background separation in selection and identification problems using boosted decision trees.
  \item Computed 95\% confidence intervals for new physical parameters using likelihoods built from data and
    statistical models of signal and background (with many nuisance parameters quantifying uncertainty).
  \item Managed Monte Carlo simulations by translating colleagues' informal requests into formal job specifications,
    testing and submitting the jobs, and monitoring the results; built tools in Python to streamline all of these steps. 
  \item Contributed 100+ pages to documents describing the experiment's sensitivity to new physics.
  \end{itemize}
}
\cventry{2004--2005}{Research Assistant}{ATLAS Experiment}{Univ. of Michigan}{Geneva, Switzerland}{
  \begin{itemize}
  \item Validated software by broadly and systematically comparing alternative systems for unexpected discrepancies;
    found and reported on important bugs.
  \item Led a team of five students to complete assembly and testing of large detector components.
  \end{itemize}
}
\cventry{1994--2004}{Misc. research and teaching}{U. of Washington \& U. of Michigan}{}{Seattle \& Ann Arbor}{
  \begin{itemize}
  \item Taught Calculus I \& II, Differential Equations, etc. at the University of Michigan
  \item Teaching assistant for Discrete Math, Computer Graphics, and Digital Design at the University of Washington
  \item Developed software simulating X-ray imaging systems and for testing balloon and satellite experimental hardware in Geophysics at the U. of Washington
  \end{itemize}
}


\section{Education}
\cventry{2011}{Ph.D. Physics (experimental, high energy)}{University of Michigan}{Ann Arbor}{}{}
\cventry{2003}{M.S. Mathematics}{University of Michigan}{Ann Arbor}{}{}
\cventry{1999}{B.S. Computer Engineering \& Mathematical Sciences (dual degree)}{University of Washington}{Seattle}{}{}

%%%%%%%%%%%%%%%%%%%%%%%%%%%%%%%%%%%%%%%%%%%%%%%%%%%%%%%%%%%%%%%%%%%%%%%%%%%%%%%
\section{Proficiencies \& technical interests}
%%%%%%%%%%%%%%%%%%%%%%%%%%%%%%%%%%%%%%%%%%%%%%%%%%%%%%%%%%%%%%%%%%%%%%%%%%%%%%%

\begin{cvcolumns}{}
  \cvcolumn{Almost every day}{
    \begin{itemize}
    \item Python, iPython \& notebooks
    \item SQL (mostly for SQL Server, Hive, Presto, and SQLite)
    \item git, standard Linux/Unix tools (shell scripting, sed, emacs/vi, etc.)
    \item basic cloud computing (mostly AWS)
    \item numpy, scipy, matplotlib
    \item pytorch, tensorflow
    \item iPython notebook
    \end{itemize}
  }
  \cvcolumn{Occasionally}{
    \begin{itemize}
    \item Airflow / Kubernettes
    \item MLFlow / DVC
    \item Statistical modeling
    \item A wide breadth of ML modeling techniques
    \end{itemize}
  }
  \cvcolumn{Dabble in}{
    \begin{itemize}
    \item Hardware, IoT
    \item HTML, CSS, javascript, jQuery, web app deployment
    \item Web scraping
    \item Mathematica, Matlab, Octave
    \item Perl, Lisp dialects
    \item (Social) network analysis
    \item Coding and compression theory
    \end{itemize}
  }
\end{cvcolumns}

\nocite{*} % <==========================================================
\bibliographystyle{unsrt}
\bibliography{AlanWilsonResume} % To use bib file created by filecontents

\end{document}
