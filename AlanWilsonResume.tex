\documentclass[11pt,letterpaper,sans]{moderncv}

\moderncvtheme{classic}
\moderncvcolor{blue}

\usepackage[utf8]{inputenc}
\usepackage[scale=0.87, top=.35in, bottom=.35in, left=.35in, right=.35in]{geometry}

% For some reason the {\color{color1} #1} version of this
% leads to space before a list that starts with \hilight
\newcommand{\hilight}[1]{\textcolor{color1}{\itshape#1}}

\firstname{Alan}
\familyname{WILSON}
%\address{1375 Civic Center Dr}{Santa Clara 95050}{}
\mobile{(408) 242-5090}
\email{alan.w.wilson@gmail.com}
\homepage{http://thingsstufftimes.com}
\extrainfo{github.com/aww \qquad @thingstimes \\ linkedin.com/pub/alan-wilson/61/91/126/}

%\nopagenumbers{}

\begin{document}
%\renewcommand{\listitemsymbol}{+}

\makecvtitle

\vspace*{-8mm}

%\section{The basics}
%\cvitem{}{As a curious human,
%  I collaborate with colleagues to identify and understand structure in data
%  and communicate our findings clearly.}

\section{Education}
\cventry{2011}{Ph.D. Experimental Particle Physics}{University of Michigan}{Ann Arbor}{}{}
\cventry{2003}{M.S. Mathematics}{University of Michigan}{Ann Arbor}{}{}
\cventry{1999}{B.S. Computer Engineering}{University of Washington}{Seattle}{}{}
\cventry{    }{B.S. Mathematical Sciences}{University of Washington}{Seattle}{(dual degree)}{}

%%%%%%%%%%%%%%%%%%%%%%%%%%%%%%%%%%%%%%%%%%%%%%%%%%%%%%%%%%%%%%%%%%%%%%%%%%%%%%%
\section{Proficiencies \& technical interests}
%%%%%%%%%%%%%%%%%%%%%%%%%%%%%%%%%%%%%%%%%%%%%%%%%%%%%%%%%%%%%%%%%%%%%%%%%%%%%%%

\begin{cvcolumns}{}
  \cvcolumn{Almost every day}{
    \begin{itemize}
    \item Python, C++, cint, ROOT
    \item git, svn, etc.
    \item numpy+scipy+matplotlib
    \item iPython notebook
    \item Standard linux tools, VMs
    \item Cluster and cloud computing
    \end{itemize}
  }
  \cvcolumn{Occasionally}{
    \begin{itemize}
    \item SQLlite and MySQL
    \item Statistical modeling
    \item ML: boosted decision trees
    \item AWS, shell scripting
    \item HTML, CSS, javascript, jQuery 
%    \item Powerpoint, Illustrator, HTML, etc.
    \end{itemize}
  }
  \cvcolumn{Dabble in}{
    \begin{itemize}
    \item D3, nltk, web scraping
    \item Mathematica, Matlab, Octave
    \item Perl, Lisp dialects
    \item (Social) network analysis
    \item Coding and compression theory
    \end{itemize}
  }
\end{cvcolumns}

%%%%%%%%%%%%%%%%%%%%%%%%%%%%%%%%%%%%%%%%%%%%%%%%%%%%%%%%%%%%%%%%%%%%%%%%%%%%%%%
%\section{Experience}
%%%%%%%%%%%%%%%%%%%%%%%%%%%%%%%%%%%%%%%%%%%%%%%%%%%%%%%%%%%%%%%%%%%%%%%%%%%%%%%

\section{Experience}
\cventry{2014--present}{Fellow at Insight Data Science}{}{}{Mountain View, CA}{
  \begin{itemize}
  \item Developed web app 'NewsSpectrum' which recommends a spectrum of alternative news articles on a similar topic.
  \item Scraped and extracted information from Google News and individual sources with scrapy, BeautifulSoup, and goose. 
  \item Used Python tools nltk and scikit-learn to tokenize, cluster, and rank articles by sophistication.
  \item Published application using flask, Bootstrap, D3, gunicorn, and supervisor on AWS.
  \end{itemize}
}
\cventry{2011--2013}{Post Doctoral Research Fellow}{ATLAS Experiment}{}{Geneva, Switzerland}{
  \begin{itemize}
  \item Filtered massive datasets (> 1 TB) down to a few important records using world-wide and local batch computing
    for the discovery of the Higgs boson.
  \item Built C++ applications (such as for the filtering above) on top of the shared tools of a collaboration of 2000 scientists.
  \item Developed a framework in Python and ROOT for efficiently specifying, building, and publishing plots to the web.
  \item Used JSON/YAML+Python to organize and run unit tests on numbers
    appearing in \LaTeX{} documents.
  %\item Constructed event visualizations in various forms
  \item Controlled and monitored data acquisition, requiring quick reactions and efficient communication with colleagues.
  \item Mentored graduate students, and was the primary editor for many documents including published papers.
  \end{itemize}
}
\cventry{2005--2010}{Graduate Student Research Assistant}{ATLAS \& D\O\ Experiments}{}{Michigan \& Illinois}{
  \begin{itemize}
  \item Used Monte Carlo simulation and data-driven methods to build statistical models of expected observations.
  \item Separated signal from background using boosted decision trees.
  \item Computed 95\% confidence intervals for new physical parameters using likelihoods built from data and
    statistical models of signal and background (with many nuisance parameters quantifying uncertainty).
  \item Managed Monte Carlo simulation by translating colleagues' informal requests into formal job specifications,
    testing and submitting the jobs, and monitoring the results; built tools in Python to streamline all of these steps. 
  \item Important contributor to large, public documents describing the experiment's sensitivity to new physics.
  \item Collaborated with engineers, technicians, and many other physicists on hardware and analysis projects.
  \end{itemize}
}
\cventry{2004--2005}{Research Assistant}{ATLAS Experiment}{Univ. of Michigan}{Geneva, Switzerland}{
  \begin{itemize}
  \item Validated software by broadly and systematically comparing alternative systems for unexpected discrepancies,
    found and reported on important bugs.
  \item Lead a team of five students to complete assembly and testing of large detector components.
  \end{itemize}
}
\cventry{1999--2004}{Misc. Teaching}{}{}{}{
  \begin{itemize}
  \item Univ. of Washington CSE Dept.:
    assisted with Discrete Math, Computer Graphics, and Digital Design.
  \item Univ. of Michigan Math Dept.:
    taught Precalculus, Calculus I and Calculus II, and assisted with Differential Equations.
  \end{itemize}
}
%% \cventry{1994--1999}{Research Assistant}{Space Sciences, Geophysics}{Univ. of Washington}{Seattle}{
%%   \begin{itemize}
%%   \item{Built \hilight{software testing platforms} for DAQ hardware used on balloon and satellite experiments}
%%   \item{\hilight{Simulated coded aperture imaging} used at X-ray wavelengths (where lenses are not possible)}
%%   \end{itemize}
%% }

%% \section{Other interests}

%% \cvitem{hobbies}{electronics, photography -- small analog and microcontroller projects, digital and chemical darkrooms}
%% \cvitem{culture}{cooking, travel, hiking, and wandering -- seeing, smelling, touching, and tasting the world}

\end{document}
